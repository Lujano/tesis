\chapter{Dataset local}
\label{capitulo6}
\lhead{Capítulo 6. \emph{Dataset local}}


\section{Introducción}
El objetivo de este capítulo es explicar la elaboración de un dataset visual inercial, a través de la construcción de un robot prototipo, la adquisición de datos de odometría de los encoders del robot, de la captura de imágenes de la cámara empleada, y de la adquisición de las medidas tomadas por la IMU.


\clearpage

\section{Descripción del prototipo}

El prototipo realizado consiste en un robot diferencial inicialmente diseñado para la categorría Sumo de la VII Competencia Nacional de Robótica CCSBOTS 2018. El prototipo tiene un peso de 2.9 kg y es capaz de soportar pesos de hasta 4kg. Posee una placa de control a bajo nivel de los motores del robot, y una Raspberry Pi 3 que es utilizada para controlar el robot utilizando un mando del tipo joystick a través de la interfaz bluetooth. El robot posee además encoders en sus ruedas que permiten tener datos de odometría. Este prototipo tiene una autonomía aproximada de 1 hora. La figura \ref{imagen:Robot} muestra la configuración final del robot. 

Más información puede ser encontrada en el \href{https://github.com/Robot-Sumo/}{repositirio de desarrollo del robot} .


\begin{figure}[H]
	\centering		\includegraphics[width=0.7\linewidth]{imagenes/prototipo/Robot}
	\caption{Prototipo final.}
	\label{imagen:Robot}
\end{figure}


\section{Diseño mecánico}


El diseño mecánico del robot está centrado en el diseño de la caja de velocidad del robot. El modelaje 3D del robot se realizó utilizando Autocad como software de diseño y visualización del prototipo.



\subsection{Motores}

Los motores empleados en el diseño del robot fueron 4 motores con escobillas Mabuchi  C2162-60006. La figura \ref{imagen:prototipo/CajaMotores} presenta la configuración utilizada. El cuadro \ref{MabuchiDatasheet} presenta la información del motor suminastrada por el fabricante y el cuadro \ref{MabuchiResultados} presenta los resultados de las mediciones realizadas al motor para la caracterización de su resistencia de armadura.

\begin{figure}[H]
	\centering
	\includegraphics[width=0.7\textwidth]{prototipo/CajaMotores}
	\caption[Motores Mabuchi empleados en el diseño del prototipo]{Componentes de Hardware del robot.}
	\label{imagen:prototipo/CajaMotores}
\end{figure}

\begin{table}[htbp]
	\caption{Datos del motor Mabuchi proveidos por el fabricante.}
	\centering
	\begin{tabular}{|l|c|}
		\hline
		\multicolumn{ 2}{|c|}{\textbf{Motor  Mabuchi Brush C2162-60006 19-24v HP}} \\ \hline
		Rating [volts] & 19 \\ \hline
		Test [volts] & 24 \\ \hline
		Stall Torque [N-cm] & 28.7 \\ \hline
		Max. Power [Watts] & 34.2 \\ \hline
		Max. Power [mili-hp] & 45.8 \\ \hline
		Duration [sec] & 30 \\ \hline
		Energy [Joules] & 1026 \\ \hline
		Weight [grams] & 224 \\ \hline
		Power/Weight [Watts/kg] & 153 \\ \hline
		Energy/Weight [Joules/kg] & 4580 \\ \hline
	\end{tabular}
	\label{MabuchiDatasheet}
\end{table}



\begin{table}[htbp]
	\caption{Resultados de mediciones realizadas al motor.}
	\centering
	\begin{tabular}{|l|c|}
		\hline
		No Load Current [Amps] & 0.15 \\ \hline
		Resistance [Ohms] & 15 \\ \hline
		Stall Current [amps] & 1.6 \\ \hline
		No Load Speed [rpm] & 3500 \\ \hline
	\end{tabular}
	\label{MabuchiResultados}
\end{table}




Entre los datos relevantes en estos cuadros destacan la corriente del motor cuando es aplicado el máximo torque (Stall Current), la cual es utilizada para el diseño de las protecciones y de los puntos de referencia de los drivers, el máximo torque aplicado por el motor (Stall Torque)  y velocidad sin carga (No Load Speed), las cuales son utilizadas para el criterio de selección de la relación de reducción en el sistema de engranajes.

\subsection{Caja de velocidad}


El cuadro \ref{TablaPruebaEngranajes} presenta los resultados estimados  de torque máximo y velocidad máxima del robot para una relación de engranajes específica. En este caso la velocidad máxima del motor fue aproximada en 3000 rpm ya que la medición sin carga fue cercana a los 3500 rpm, y estas revoluciones bajan al presentar carga en el sistema de transferencia. El torque máximo inicial representa el aproximado de la suma del torque máximo de dos motores, cuyo valor se tomó cercano a 30 N-cm, respecto al valor de de 28.7 N-cm proveido por el fabricante en el cuadro \ref{MabuchiDatasheet}.

\begin{table}[htbp]
	\caption{Pruebas de diferentes relaciones de engranajes}
	\begin{tabular}{|l|r|r|r|r|}
		\hline
		\textbf{Relación final de engranajes} & \textbf{20} & \textbf{15} & \textbf{12} & \textbf{8} \\ \hline
		Velocidad máxima del motor (rpm) & 3000 & 3000 & 3000 & 3000 \\ \hline
		Radio de la rueda (cm) & 3.74 & 3.74 & 3.74 & 3.74 \\ \hline
		Velocidad máxima de la rueda (rpm) & 150 & 200 & 250 & 375 \\ \hline
		Velocidad máxima de la rueda (rev/s) & 15.708 & 20.944 & 26.18 & 39.27 \\ \hline
		Torque máximo del Motor (N-cm) & 60 & 60 & 60 & 60 \\ \hline
		\textbf{Torque máximo final (N-cm)} & \textbf{1200} & \textbf{900} & \textbf{720} & \textbf{480} \\ \hline
		\textbf{Velocidad máxima del robot (m/s)} & \textbf{0.59} & \textbf{0.78} & \textbf{0.98} & \textbf{1.47} \\ \hline
	\end{tabular}
	\label{TablaPruebaEngranajes}
\end{table}



Con los resultados obtenidos para diferentes relaciones de engranajes, se utilizó Unity para simular las propiedades dinámicas del robot y de esta manera seleccionar la relación de engranajes con la mejor relación torque-velocidad. En la figura \ref{imagen:prototipo/SimulacionUnity} se presenta una captura de esta simulación, en la cual el robot debía ser capaz de mover a un robot con un peso de 3kg con características similares.



\begin{figure}[H]
	\centering
	\includegraphics[width=0.7\textwidth]{imagenes/prototipo/SimulacionUnity}
	\caption{Simulación en Unity de las características dinámicas del robot para la elección de la relación de reducción de velocidad.}
	\label{imagen:prototipo/SimulacionUnity}
\end{figure}

La relación final de reducción seleccionada fue de  15:1. De esta forma, el sistema de engranajes fue diseñado para cumplir esta relación. La figura \ref{imagen:disenoEngranajesAutocad} presenta el diseño realizado. En éste, se tiene un engranaje de 20 dientes acoplado a un engranaje de 60 dientes, el cual se encuentra acoplado en su mismo eje a un engranaje de 20 dientes. Posterioremente se tiene un engranajes de transmision de 20 dientes que se encarga de transmitir la velocidad al engranaje de la rueda, el cual fue diseñado para tener 100 dientes. Este engranaje es el de mayor tamaño en la figura \ref{imagen:disenoEngranajesAutocad}. De esta forma se garantizó la relación de reducción a 15.

\begin{figure}[H]
	\centering
	\includegraphics[width=0.7\linewidth]{imagenes/prototipo/EngranajesAutocad}
	\caption{Parte del modelado en 3D del robot realizado en Autocad}
	\label{imagen:disenoEngranajesAutocad}
\end{figure}


Las imágenes oresentadas en la figura \ref{imagen:construccionFinalRobot} presentan parte del proceso de construcción de la caja de velocidades. Los engranajes presentados fueron impresos utilizando el diseño 3D previamente realizado, y utilizando como material de impresión el polímero PLA. Cabe acotar que el empleo de dos motores por cada sistema de engranajes fue realizado para aumentar el torque de la caja de velocidad.

\begin{figure}[H]
	\centering		\includegraphics[width=0.7\linewidth]{imagenes/prototipo/Motores}
	\includegraphics[width=0.4\linewidth]{imagenes/prototipo/CajaColocandoEngranejes}
	\includegraphics[width=0.4\linewidth]{imagenes/prototipo/CajaLista}

	\caption{Imagenes de armado del robot}
	\label{imagen:construccionFinalRobot}
\end{figure}



\subsection{Drivers}

Los drivers empleados en el diseño del robot fueron los STA6940M, los cuales soportan motores con escobillas con voltajes de operación de hasta 44V y 4 amperios de corriente promedio, y son compatibles con la tensión de alimentación lógica de 5V. La figura \ref{imagen:Driver} presenta el empaquetado 18-pin ZIP del driver empleado.

\begin{figure}[H]
	\centering		\includegraphics[width=0.3\linewidth]{imagenes/prototipo/Driver}
	\caption{Driver STA6940M. }
	\label{imagen:Driver}
\end{figure}

La configuración utilizada en el robot es similar a la recomendada por el fabricante en la figura \ref{imagen:DriverAplicacion}. El diagrama de pines se muestra en la tabla \ref{imagen:DriverTabla}.

\begin{figure}[H]
	\centering		\includegraphics[width=0.7\linewidth]{imagenes/prototipo/InformacionDeAplicacion}
	\caption{Diagrama de aplicación del driver STA6940M proveído por el fabricante.}
	\label{imagen:DriverAplicacion}
\end{figure}


\begin{figure}[H]
	\centering		\includegraphics[width=0.7\linewidth]{imagenes/prototipo/TablaDePines}
	\caption{Lista de pines del driver STA6940M proveída por el fabricante. }
	\label{imagen:DriverTabla}
\end{figure}

Cabe destacar que la protección de sobrecorriente que presenta el driver ($OCP\_ REF$) es colacada manualmente con un arreglo de resisitencias ,al igual que la referencia del PWM ($PWM\_ REF$). En particular, el voltaje de $OCP\_ REF$ debe ser igual al voltaje de la resistencia de sensado cuando el motor presenta su punto de máximo torque, que en este caso es cuando se alcanza la corriente de Stall a 1.6 Amperios. En la configuración final se utilizan dos motores en paralelo para cada driver por lo que el ajuste se realizó para 3.2 Amperios.


\subsection{Encoders}

El sistema de encoders también fue diseñado desde cero. La figura \ref{imagen:RuedaEncoders} presenta el modelaje 3D de la rueda del encoder y la disposición final sobre el eje de la rueda del robot.

\begin{figure}[H]
	\centering		\includegraphics[width=0.3\linewidth]{imagenes/prototipo/Ruedaencoder}
	\includegraphics[width=0.3\linewidth]{imagenes/prototipo/EncoderReal}
	\caption{Diseño 3D de la rueda de los encoders del robot}
	\label{imagen:RuedaEncoders}
\end{figure}

El diseño final de la rueda del encoder tiene 40 ranuras. El proceso de calibración del encoder pasó por ajustar las resistencias de los fotoemisores y fotoreceptores del encoder y posteriormente realizar una restricción del mínimo tiempo de interrupción entre dos ranuras basada en la máxima velocidad de la rueda.

Cabe destacar que los encoders presentan un circuito de adecuación para generar la interrupciones en el microcontrolador de la placa de control a bajo nivel, el cual consiste en un comparador LM311.

Los resultados finales de calibración fueron bastantes satisfactorios, obteniéndose un error de lectura de 4 ranuras por cada 4000 ranuras, el cual representa un error de 0.1\%.
Cabe destacar que la correcta lectura de los encoders es de vital importancia para obtener datos confiables de odometría del robot.

\subsection{Placa de control a bajo nivel}
Los drivers del motor, los circuitos de acondicionamiento de los encoders y el microcontrolador encargado de realizar el control PWM del robot fueron integrados en una PCB, la cual se diseño utilizando Kicad 5.0.2 para Ubuntu. La figura \ref{imagen:PlacaMotores} presenta el esquemático de la placa. El microcontrolador utilizado fue el Arduino Nano 328p.


\begin{figure}[H]
	\centering		\includegraphics[width=1.0\linewidth]{imagenes/prototipo/Placa/PlacaMotores}
	\caption{Esquemático de la placa de los motores.}
	\label{imagen:PlacaMotores}
\end{figure}

Las figura \ref{imagen:PlacaMotoresFrontal} presenta la vista frontal de las capas de la placa, su vista 3D y el modelo final.


%\begin{figure}[H]
%	\centering		
%	\includegraphics[width=0.7\linewidth]{imagenes/prototipo/Placa/3dViewerBottom}
%	\caption{Vista trasera de las capas de la placa de los motores}
%	\label{imagen:PlacaMotoresBottom}
%\end{figure}


\begin{figure}[H]
	\centering	
	\includegraphics[width=0.7\linewidth]{imagenes/prototipo/Placa/PCB_FrontAllLayers}	\includegraphics[width=0.7\linewidth]{imagenes/prototipo/Placa/3dViewerFront}
	\includegraphics[width=0.7\linewidth]{imagenes/prototipo/Placa/PCB_FinalFront}
	\caption{Vista frontal de la placa de los motores.}
	\label{imagen:PlacaMotoresFrontal}
\end{figure}

\subsection{Sistema de alimentación y autonomía}

El robot es alimentado por dos paquetes de baterías de 12V, los cuales fueron construidos con baterías de litio Sony G5 18650 de 2200mAh. Cada paquete dispone de 6 baterias de litio, para un total de 12 baterías.  Estas baterías se disponen en serie para lograr el voltaje de 24V de funcionamiento de los motores del robot, con una capacidad de 4400mAh. Se aprovecha la toma de 12V para alimentar la placa de control de los motores y la raspberry utilizando dos conversores del tipo Buck independientes. Los conversores empleados son del tipo MP2307, y su voltaje de salida es fijado a  5V.
La figura \ref{imagen:BuckCurva} presenta la curva de eficiencia del conversor utilizado.


\begin{figure}[H]
	\centering	
	\includegraphics[width=0.5\linewidth]{imagenes/prototipo/Buck}
	\caption{Curva de eficiencia vs corriente de carga del conversor Buck utilizado}
	\label{imagen:BuckCurva}
\end{figure}


En la figura \ref{imagen:Bateria} se presenta el proceso de carga de uno de los paquetes de batería de 12V. En este caso se utilizan 3 circuitos de cargadores de batería de litio de 3.7V HW-107 que emplean el controlador de carga TC4056. Estos circuitos se encuentran eléctricamente aislados ya que utilizan cargadores independientes de la red eléctrica nacional a 5V@1A. El tiempo aproximado de carga de cada paquete de batería es de 6 horas.

\begin{figure}[H]
	\centering	
	\includegraphics[width=0.7\linewidth]{imagenes/prototipo/Bateria}
	\caption{Carga del paquete de baterías de 12V del robot.}
	\label{imagen:Bateria}
\end{figure}

También se presenta el cálculo de autonomía del robot en función de la eficiencia de los conversores tipo Buck y del consumo de los componentes del robot.

La tabla \ref{Buck1} presenta el consumo de carga promedio del conversor Buck 1. En este caso la corriente de carga aproximada es de 115.2 mA, y debido a que el voltaje de salida es de 5V, la potencia de consumo es 0.58 W. En esta caso se aproxima la eficiencia del conversor a 90\% para el cálculo de las perdidas de conversión. La tabla \ref{Buck1Perdidas} presenta la corriente de alimentación suminastrada por el arreglo de baterías de 12V en el proceso de conversión, la cual es la potencia total consumida por el conversor entre 12V.

\begin{table}[htbp]
	\caption{Consumo de carga del conversor Buck 1}
	\begin{tabular}{|l|c|c|c|c|}
		\hline
		\multicolumn{1}{|c|}{\textbf{Componente}} & \textbf{Cantidad} & \textbf{ Vin (V)} & \textbf{ Ipr (mA)} & \textbf{Itotal (mA)} \\ \hline
		Arduino Nano 328p & 1 & 5 & 15 & 15 \\ \hline
		Leds PCB & 3 & 5 & 10 & 30 \\ \hline
		Fotoreceptor-Fotoemisor Encoder & 2 & 5 & 10 & 20 \\ \hline
		Comparador LM311 & 2 & 5 & 5.1 & 10.2 \\ \hline
		Driver STA6940M & 2 & 5 & 20 & 40 \\ \hline
		& \multicolumn{1}{l|}{} & \multicolumn{1}{l|}{} & \textbf{Iload (mA)} & 115.2 \\ \hline
	\end{tabular}
	\label{Buck1}
\end{table}



\begin{table}[htbp]
	\caption{Consumo conversor Buck 1}
	\begin{tabular}{|l|c|}
		\hline
		Consumo Buck 1 (W) & 0.58 \\ \hline
		Perdidas por Conversión (W) & 0.06 \\ \hline
		Consumo Total (W) & 0.64 \\ \hline
		Corriente de Alimentación Equivalente (mA) & \multicolumn{1}{r|}{53.3333333333} \\ \hline
	\end{tabular}
	\label{Buck1Perdidas}
\end{table}


En el caso del consumo del conversor Buck 2, se calcula  en función del consumo promedio de la Raspberry Pi 3 utilizando sus cuatro núcleos al 100\%, el consumo de la cámara y el de la IMU. La tabla \ref{} presenta estos resultados.

\begin{table}[htbp]
	\caption{Consumo conversor Buck 2}
	\begin{tabular}{|l|c|c|c|c|}
		\hline
		\multicolumn{1}{|c|}{\textbf{Componente}} & \textbf{Cantidad} & \textbf{ Vin (V)} & \textbf{ Ipr (mA)} & \textbf{Itotal (mA)} \\ \hline
		IMU MPU6050 & 1 & 3.3 & 3.8 & 3.8 \\ \hline
		Raspberry Pi 3  & 1 & 5 & 730 & 730 \\ \hline
		Pi Camera  v1.2 & 1 & 1.3 & 250 & 250 \\ \hline
		& \multicolumn{1}{l|}{} & \multicolumn{1}{l|}{} & \textbf{Iload (mA)} & 983.8 \\ \hline
	\end{tabular}
	\label{Buck2}
\end{table}

\begin{table}[htbp]
	\caption{Consumo conversor Buck 2}
	\begin{tabular}{|l|c|}
		\hline
		Consumo Buck 2 (W) & 1.3 \\ \hline
		Perdidas por Conversión (W) & 0.54 \\ \hline
		Consumo Total (W) & 5.4 \\ \hline
		Corriente de Alimentación Equivalente (mA) & 450 \\ \hline
	\end{tabular}
	\label{Buck2Perdidas}
\end{table}

Por tanto el consumo de corriente total de los Bucks es de  503mA, a lo cual se le debe sumar los 250mA estimados que consume en promedio cada motor del robot. Por tanto el consumo total del robot es de 1.5A, por paquete de baterías. Como cada paquete presente una capacidad de 4400 mAh, la autonomía estimada es de aproximadametne 2 horas. Sin embargo, la autonomía real del robot es de aproximadamente 1 hora debido a que las baterías empleadas son recicladas.

\subsection{Joystick}




\begin{figure}[H]
	\centering		\includegraphics[width=0.7\linewidth]{imagenes/prototipo/Raspberry}
	\caption{Disposición de sensores en la Rasperry Pi 3}
	\label{imagen:Raspbery}
\end{figure}

En cuanto al dataset local realizado, también fue implementado utilizando C++ como lenguaje de programación, y utilizando una raspberry pi 3 para la adquisición de los datos de la imu y de la cámara, y utilizando 4 hilos de ejecución.


\section{Componentes de hardware}

\begin{figure}[H]
	\centering
	\includegraphics[width=0.7\textwidth]{HardwareRobot}
	\caption[Componentes de Hardware del robot]{Componentes de Hardware del robot.}
	\label{imagen:HardwareRobot}
\end{figure}



Con el empleo de la IMU es posible obtener los cambios de orientación de la cámara y de su aceleración, siempre y cuando se encuentren sincronizadas las mediciones. Bajo esta restricción se tiene el residual de orientación de la IMU (${R}_{RES/IMU}$) , el cual representa el cambio de orientación de la IMU entre el frame actual y el anterior, tal como se presenta en la ecuación \ref{eq:residualIMU}.

\begin{table}[htbp]
	\caption{Presupuesto del robot prototipo}
	\begin{tabular}{|l|c|c|c|}
		\hline
		\multicolumn{1}{|c|}{\textbf{Componente}} & \textbf{Cantidad} & \textbf{Valor (\$)} & \textbf{Precio Total (\$)} \\ \hline
		Arduino Nano 328p & 1 & 1.9 & 1.9 \\ \hline
		Batería Litio Sony G5 18650  2200mAh  & 12 & 2.0 & 24.0 \\ \hline
		Caja Acrílico Robot Diferencial & 1 & 25.0 & 25.0 \\ \hline
		Case Raspberry Pi 3 & 1 & 2.0 & 2.0 \\ \hline
		Caster & 1 & 4.0 & 4.0 \\ \hline
		Comparador LM311 & 2 & 0.1 & 0.2 \\ \hline
		Conectores y Cableado & 1 & 2.0 & 2.0 \\ \hline
		Driver STA6940M & 2 & 5.0 & 10.0 \\ \hline
		Ejes, Rolineras, Retenedores, Tornillos & 1 & 8.0 & 8.0 \\ \hline
		Fotoreceptor-Fotoemisor Impresora & 2 & 0.2 & 0.4 \\ \hline
		Frente Metálico & 1 & 1.0 & 1.0 \\ \hline
		Fusible 1A  Modelo  Americano & 2 & 0.1 & 0.2 \\ \hline
		Fusible 350 mA Modelo Europeo & 1 & 0.1 & 0.1 \\ \hline
		Impresión Engranajes y Rueda Encoder & 1 & 2.0 & 2.0 \\ \hline
		IMU MPU6050 & 1 & 0.6 & 0.6 \\ \hline
		Joystick PS3 & 1 & 10.0 & 10.0 \\ \hline
		Mini DC Buck Converter MP2307 & 2 & 0.4 & 0.8 \\ \hline
		Motor Mabuchi C2162-60006 19-24v Hp & 4 & 3.5 & 14.0 \\ \hline
		PCB Control Bajo Nivel & 1 & 10.0 & 10.0 \\ \hline
		Pi Camera  v1.2 & 1 & 7.2 & 7.2 \\ \hline
		Raspberry Pi 3 & 1 & 35.0 & 35.0 \\ \hline
		Rueda de Goma 7.48 mm de Diámetro & 2 & 4.0 & 8.0 \\ \hline
		& \multicolumn{1}{l|}{\textbf{Total (\$)}} & \multicolumn{1}{l|}{} & \textbf{166.44} \\ \hline
	\end{tabular}
	\label{PresupuestoRobot}
\end{table}



\section{Componentes}





